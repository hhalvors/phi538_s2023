\documentclass[11pt]{article}
\usepackage{url}
\usepackage{hyperref}
\usepackage{fullpage}
\usepackage{enumitem}
\setlist[itemize]{itemsep=0.1em}
\newcommand\rurl[1]{%
  \href{http://#1}{\nolinkurl{#1}}%
}

\usepackage{graphicx}
\def\rot#1{\rotatebox[origin=c]{180}{#1}}

\usepackage{doi}
\usepackage[backend=biber,natbib=true,style=authoryear]{biblatex}
\addbibresource{relativity-lessons.bib}


\begin{document}

\section*{Philosophy of Physics: Spacetime and Objectivity}

% Zuordnungssystem

%% one way speed of light 

% what's special about diffeomorphisms?

%% Galilean spacetime
%% https://ncatlab.org/zoranskoda/show/Galilean+structure
%% Arnold. Mathematical methods classical mechanics
%% Penrose, Road to reality (fiber bundle)

%% Epstein. The Geometrical Language of Continuum Mechanics
%% Tausk. Notes on Mathematical Physics for Mathematicians
% Dodson and Poston. Tensor geometry
% Emam. Covariant Physics: From Classical Mechanics to General Relativity and Beyond

%% definability

%% concept empiricism

%% Mundy dissertation (synthetic affine geometry)

%% passive versus active transformations 

%% Poincare group

%% Ehrenfest paradox

%% synthetic differential geometry

%% Einstein algebras 

%% invariants

%% Eric Mascall

Spring 2023

\medskip \noindent This seminar will run in two intertwining
streams. In the first stream, we will learn/review spacetime geometry
(with focus on Minkowski spacetime, and including some basic material
on Lorentzian manifolds). In the second stream, we will discuss the
relevant philosophical issues. (Less scientifically oriented students
can follow the second stream without paying much attention to the
first.)  To be more concrete, we spend week 1 talking about general
philosophical issues. Then we spend weeks 2--4 getting the basics of
spacetime geometry under our belt. From then on (weeks 5--12), we
spend one third of each session on spacetime geometry, and two thirds
on discussion of the philosophical issues.

% \subsection*{Grading scheme for UG students}

% \begin{itemize}
% \item Weekly problem sets (60\%) Due at the beginning of seminar each
%   week. Problems will be drawn primarily from the two Malament books
%   below
% \item In-class presentation (20\%) Each student will sign up to
%   present and lead discussion during one of the seminar sessions
% \item Take-home final exam (20\%) Due on Dean's Date
% \end{itemize}

\subsection*{Spacetime geometry}

Our two main sources are:
\begin{enumerate}
\item Malament. \emph{Notes on geometry and spacetime}
  \url{philsci-archive.pitt.edu/16760/}
\item Malament. \textit{Topics in the foundations of general
    relativity and Newtonian gravitation theory}
\end{enumerate}

% Penrose, The road to reality

\subsection*{Philosophical issues}

\begin{enumerate}
\item Absolute vs Relative $|$ Objective vs Subjective
  \begin{itemize}
  \item Selections from Bernard Williams, \emph{Descartes: the project
      of pure enquiry} and \emph{Ethics and the limits of philosophy}
  \item Putnam. ``Bernard Williams and the absolute conception of the
    world'' in \emph{Renewing philosophy}
  \item Holt. Invariance \rurl{edge.org/response-detail/27053}
  \end{itemize}
  % invariants

\item Issues in the philosophy of time  

\item Paradox? Twins, barns, and distant stars % Andromeda: Emperor's new mind
  \begin{itemize}
  \item McCall and Lowe. 3D/4D equivalence, the twins paradox and
    absolute time \rurl{doi.org/10.1111/1467-8284.00020}
  \item Nerlich. How the twins do it: STR and the clock paradox
    \rurl{doi.org/10.1093/analys/64.1.21}
  \item Miller. The twins’ paradox and temporal passage
    \rurl{doi.org/10.1111/j.0003-2638.2004.00486.x}
  \item Friebe. Twins' paradox and closed timelike curves: the role
    of proper time and the presentist view on spacetime
    \rurl{doi.org/10.1007/s10838-012-9194-0}
  \item Maudlin. \textit{Philosophy of physics: space and time}
  \item Barrow and Levin. Twin paradox in compact spaces
  \item Weeks. The twin paradox in a closed universe
  \end{itemize}
  %% Penrose Andromeda paradox



%% TO DO: Johnston -- we don't know what we mean by "the present"  

\item Does relativity theory refute objective becoming? 
  \begin{itemize}
  \item Putnam. Time and physical geometry
  \item Stein. On Einstein-Minkowksi space-time
  \item Sider. \emph{Four dimensionalism}
  \item Balashov and Janssen. Critical notice: presentism and
    relativity
  \item Craig. ``The elimination of absolute time by the special
    theory of relativity'' in \textit{God and time: essays on the
      divine nature}
  \item Craig. ``The metaphysics of special relativity: three views''
  \item Craig. \textit{Time and the metaphysics of relativity}
  \item Zimmerman. Presentism and the spacetime manifold
  \item Pooley. Relativity, the open future, and the passage of time
  \item Rovelli. Neither presentism nor eternalism
    \rurl{doi.org/10.1007/s10701-019-00312-9}
  \item Dorato. Putnam on time and special relativity: a long journey
    from ontology to ethics
  \item Eagle. Relativity and the A-theory
    \rurl{doi.org/10.4324/9781315623818-11}
  \item Thyssen. The Rietdijk-Putnam-Maxwell argument
  \end{itemize}


% Skow shapes intrinsic?
  
\item Four-dimensional objects
  \begin{itemize}
  \item Costa et al. Relativity and three four-dimensionalisms
    \rurl{doi.org/10.1111/phc3.12308}
  \item Balashov. Relativistic objects \rurl{doi.org/10.1111/0029-4624.00198}
  \item Davidson. Special relativity and the intrinsicality of
    shape \rurl{doi.org/10.1093/analys/ant100}
  \item Balashov. On the invariance and intrinsicality of
    four-dimensional shapes in special relativity
  \item Calosi. The relativistic invariance of 4D shapes
  \item Halvorson. Invariance and ontology in relativistic physics
    \rurl{philpapers.org/rec/HALTIN-3}
 \item Penrose. The apparent shape of a relativistically moving
    sphere
  \item Terrell. Invisibility of the Lorentz contraction
  \item K{\"o}lbel. Objectivity and perspectival content
    \doi{10.1007/s10670-019-00188-1}
  \item Le Bihan. From spacetime to space and time: A reply to
    Markosian \doi{10.1093/analys/anz098}
  \end{itemize}

\item Fine's fragmentalism
  \begin{itemize}
  \item Fine. ``Tense and reality'' in \emph{Modality and Tense}
    \rurl{doi.org/10.1093/0199278709.001.0001}
  \item Lipman. On Fine's fragmentalism
  \item Hofweber and Lange. Fine’s fragmentalist interpretation of special relativity
  \item Lipman. On the fragmentalist interpretation of special relativity
  \item Hofweber and Lange. Fragmentalism and special relativity
  \item Slavov. Eternalism and perspectival realism about the `now'\,
    \rurl{doi.org/10.1007/s10701-020-00385-x}
  \end{itemize}

\item Is Lorentz contraction a physical effect?
  % the rod in itself
  % is it Cambridge change? | intrinsic
  % is it change in description?
  \begin{itemize}
  \item Dieks. The `reality' of the Lorentz contraction
    \rurl{jstor.org/stable/25170686}
  \item Lange. ``How to explain the Lorentz transformations'' in
    Mumford and Tugby (eds.) \textit{Metaphysics and Science}
    \rurl{doi.org/10.1093/acprof:oso/9780199674527.003.0004}
  \item Maudlin. \textit{Philosophy of physics: space and time}
  \item \fullcite{giovanelli2023}
  \item \fullcite{epstein2018}  
  %% Favrholdt  
  \end{itemize}
  
\item Bell's Lorentzian pedagogy 
  \begin{itemize}
  \item Bell. ``How to teach special relativity''
  \item Nerlich. Bell's `Lorentzian pedagogy': a bad education
    \rurl{philsci-archive.pitt.edu/5454/1/Bell.pdf}
  \item Maudlin. ``The physics of measurement'' in \emph{Philosophy of
      Physics: Space and Time}
  \item McDonald. The relativity of acceleration
    \rurl{kirkmcd.princeton.edu/examples/rel_accel.pdf}
  \end{itemize}

\item Principle vs Constructive Theories
  \begin{itemize}
  \item Brown. \emph{Physical relativity: Spacetime structure from a
      dynamical perspective}
  \item Brown and Pooley. Minkowski spacetime: A glorious non-entity
  \item Janssen. Drawing the line between kinematics and dynamics in
    special relativity
  \item Norton. Why constructive relativity fails
  \item Lange. Did Einstein really believe that principle theories are
    explanatorily powerless?
  \item Felline. Scientific explanation between principle and
    constructive theories
  \item Giovanelli. `Like thermodynamics before Boltzmann'. On the
    emergence of Einstein's distinction between constructive and
    principle theories
  \item Giovanelli. Relativity theory as a theory of principles. A
    reading of Cassirer's Zur Einstein'schen Relativitätstheorie
  \item J. Read. Geometrical constructivism and modal relationalism:
    Further aspects of the dynamical/geometrical debate
    \doi{10.1080/02698595.2020.1813530}
  \end{itemize}

\item Conventionalism vs cutting nature at the joints

  %% hinge principles ?

  The debate about conventionalism is a very old one, and it received
  a sharp formulation after the development of non-euclidean
  geometries in the nineteenth century. The most notorious geometric
  conventionalist was Henri Poincar{\'e} who said that the choice of
  geometry cannot and should not intend to mirror the objective
  structure of space. The choice of geometry is conventional.

  Poincar{\'e}'s geometric conventionalism was taken over by the
  logical positivists and turned into a general doctrine about the
  role of analytic principles in scientific knowledge.

  By the looks of it, conventionalism as a philosophical stance was
  eradicated by the 1980s, as a result of trenchant critiques by
  Quine, Putnam, Earman, Friedman, Nerlich, etc. So why take this
  subject back up again?  In short, it's unclear what is supposed to
  replace conventionalism if not the extreme (and absurd) view that
  the form/content distinction has been abolished.
  \begin{itemize}
  \item philosophy of geometry
    %% to do: student paper - what Poincare said about winning objectivity
    %% Burgess: rigor and structure
    %% Lewis: new work theory of universals
    \begin{itemize}
    \item Helmholtz
    \item Poincar{\'e}. On the foundations of geometry
    \item Poincar{\'e}. Non-euclidean geometry and physics
    \item Heis. The geometry behind Poincar{\'e}'s conventionalism
    % Folina. Poincaré's Conception of the Objectivity of Mathematics  
    \end{itemize}
  \item Kamlah. Hans Reichenbach's relativity of
    geometry. \doi{10.1007/bf00485877}
  \item Nerlich. \textit{The shape of space}, pp 160--177
  \item Sider. \emph{Writing the book of the world}, pp 40--43
  \item Putnam. An examination of Gr{\"u}nbaum's philosophy of
    geometry \doi{10.1017/CBO9780511625268.008}
  \item Putnam. The refutation of conventionalism
    \doi{10.2307/2214643}
  \item Coffa. Geometry and semantics: an examination of Putnam's
    philosophy of geometry \doi{10.1007/978-94-009-7055-7_1}
  \item \emph{Defending Einstein: Hans Reichenbach's writings on
      space, time and motion}
  \item D{\"u}rr and Ben-Menahem. Why Reichenbach wasn't entirely
    wrong, and Poincar{\'e} was almost right, about geometric
    conventionalism
  \item DiSalle. Conventionalism and modern physics: a re-assessment
    \doi{10.1111/1468-0068.00367}
  \item Weatherall and Manchak. The geometry of conventionality
    \doi{10.1086/675680}
  \item Sklar. \textit{Space, time, and spacetime}, pp 88--147
  \item Toretti. \textit{Relativity and geometry}, pp 230--246
  \item Friedman. \textit{Foundations of space-time theories}, pp
    264--339
  \item Friedman. Poincar{\'e}'s conventionalism and the logical
    positivists
  \item Norton. ``Why geometry is not conventional: the verdict of
    covariance principles'' in \textit{Semantical Aspects of Spacetime
      Theories}
  \item Warren. \textit{Shadows of syntax: revitalizing logical and
      mathematical conventionalism}
  \item Ben-Menahem. Convention: Poincar{\'e} and some of his critics
    \rurl{jstor.org/stable/3541926}
  \item Ben-Menahem. \textit{Conventionalism: from Poincar{\'e} to
      Quine}
  \item Zahar. Poincar{\'e}'s philosophy of geometry, or does
    geometric conventionalism deserve its name?
    \doi{10.1016/S1355-2198(96)00027-5}
  \item Creath. Carnap's conventionalism
    \rurl{jstor.org/stable/20117711}
  \item Stump. Defending conventions as functionally a priori
    knowledge
  \item Bland. An analysis of conventionalism in early analytic
    philosophy. Phd Thesis, Western University 2009.
  \item D{\"u}rr and Read. Reconsidering Conventionalism: An
    Invitation to a Sophisticated Philosophy for Modern (Space-) Times
  \item Tasdan and Th{\'e}bault. Spacetime conventionalism revisited
    \doi{10.1017/psa.2023.103}
  \end{itemize}

\item Coordinative definitions $|$ Correspondence rules

  \begin{itemize}
  \item Shapiro. `Coordinative definition' and Reichenbach's semantic
    framework: A reassessment \rurl{doi.org/10.1007/BF01130757}
  \end{itemize}  

\item Relativity and the verification criterion of meaning
  %% Berkeley
  %% the aether -- unobservable structure
  %% Newtonian preferred frame
  %% https://sites.pitt.edu/~jdnorton/teaching/HPS_0410/chapters/significance_2/index.html
  %% W.L. Craig on the flaws of STR
  %% Reichenbach
  \begin{itemize}
  \item Mach  
  \item Nerlich. On learning from the mistakes of the positivists
    \rurl{doi.org/10.1016/S0049-237X(08)70060-0}
  \item Friedman. ``Relativity theory and logical positivism'' in
    \emph{Foundations of space-time theories}, pp 3--31
    % Holton, "Mach, Einstein, and the search for reality"
  \item P. Frank. ``Einstein, Mach, and logical positivism'' in
    \textit{Albert Einstein: Philosopher-Scientist}
    \url{dropbox.com/s/iridamr5gz7k67k/frank-einstein.pdf?dl=0}
  \item Norton. ``How Hume and Mach helped Einstein find special
    relativity'' in Friedman et al. (eds.) \textit{Discourse on a new
      method: reinvigorating the marriage of history and philosophy of
      science}
  \item Slavov. Time as an empirical concept in special relativity
    \rurl{doi.org/10.1353/rvm.2019.0084}
  \end{itemize}  
  
\item Frames of reference $|$ Contexts $|$ Indexicals
  \begin{itemize}
  \item DiSalle. Space and time: inertial frames
    \rurl{plato.stanford.edu/entries/spacetime-iframes}
  \item DiSalle. Conventionalism and the origins of the inertial frame
    concept \rurl{jstor.org/stable/193063}
  \item Pinillos. Time dilation, context and relative truth
  \item Kaplan. ``Demonstratives''
  \item K{\"o}lbel. Indexical relativism versus genuine relativism
  \item Mühlhölzer. ``Objektivität und Relativität'' in Weingartner
    and Czermak (eds.) \textit{Epistemology and philosophy of science}
    \rurl{dropbox.com/s/4dcc7e3djw5g0sp/muehlhoelzer.pdf}
\item Mühlhölzer. On objectivity \doi{10.1007/bf00166443}
  % what does "preferred" mean?
  % \item Solomyak. ``Temporal ontology and the metaphysics of perspectives''  
\end{itemize}

% relationship between frame of reference, coordinates, context

\item Coordinates $|$ Invariants $|$ Covariance
  \begin{itemize}
  \item Pooley. Background independence, diffeomorphism invariance and
    the meaning of coordinates
  \item Norton. Coordinates and covariance: Einstein's view of
    space-time and the modern view \rurl{doi.org/10.1007/bf00731880}
  \item Giovanelli. Nothing but coincidences: the point-coincidence
    and Einstein's struggle with the meaning of coordinates in
    physics \rurl{doi.org/10.1007/s13194-020-00332-7}
  \item Wallace. Who's afraid of coordinate systems? An essay on
    representation of spacetime structure
    \url{doi.org/10.1016/j.shpsb.2017.07.002}
  \item North. \emph{Physics, structure, and reality}
    \rurl{doi.org/10.1093/oso/9780192894106.001.0001}
  \item Barrett. Coordinates, structure, and classical mechanics: a
    review of Jill North’s Physics, Structure, and Reality
    \doi{10.1017/psa.2022.27}
  \item Suppes. Invariance, symmetry and meaning
    \rurl{doi.org/10.1023/a:1026437914611}
  \item Earman. Covariance, invariance and the equivalence of
    frames \rurl{doi.org/10.1007/BF00712691}
  \item Vollmer. Invariance and objectivity
    \rurl{doi.org/10.1007/s10701-010-9471-x}
  \item Winnie. ``Invariants and objectivity: A theory with
    applications to relativity and geometry''  
  \item Scheibe. ``Invariance and covariance'' in \textit{Space, time, and mechanics}
    \rurl{dropbox.com/s/mhd5tqhneut1laf/scheibe-invariance.pdf?dl=0}
  \end{itemize}

  % geodesic coordinates
  
\item Absolute objects
  \begin{itemize}
  \item Anderson. Covariance, invariance, and equivalence: A
    viewpoint \rurl{doi.org/10.1007/BF02450447}
  \item Friedman. \emph{Foundations of space-time theories}
    \rurl{doi.org/10.1515/9781400855124}
  \item Friedman. Relativity principles, absolute objects and symmetry
    groups \rurl{doi.org/10.1007/978-94-010-2686-4_14}
  \item Read. ``Geometric objects and perspectivalism'' in Read and
    Teh (eds.) \textit{The philosophy and physics of Noether's
      theorems} \rurl{doi.org/10.1017/9781108665445.011}
  \item Howard. ``Point coincidences and pointer coincidences:
    Einstein on invariant structure in spacetime theories'' in
    \textit{The expanding worlds of general relativity}
  \end{itemize}

\item Intrinsic versus Extrinsic
  \begin{itemize}
  \item Friedman. \textit{Foundations of space-time theories}, p 339
  \item North. \emph{Physics, structure, and reality}
    \rurl{doi.org/10.1093/oso/9780192894106.001.0001}
  \item Jacobs. Invariance, intrinsicality, and perspicuity
    \doi{10.1007/s11229-022-03682-2}
  \item Skow. Are shapes intrinsic? \doi{10.1007/s11098-006-9009-4}
  \item Marshall and Weatherson. Intrinsic vs. extrinsic properties
    \rurl{plato.stanford.edu/entries/intrinsic-extrinsic}
  \item Bader. Towards a hyperintensional theory of intrinsicality
    \doi{10.5840/jphil2013110109}
  \end{itemize}

\item Substantivalism vs Relationalism
  \begin{itemize}
  \item Newton. \textit{De Gravitatione}
  \item Leibniz-Clarke correspondence
  \item Berkeley. \textit{Philosophical writings}
  \item Huggett. Motion and relativity before Newton
  \item Maudlin. ``Space, absolute, and relational'' in Le Poidevan
    (ed.) \textit{The Routledge companion to metaphysics}
  \item Pooley. ``Substantivalist and relationalist approaches to
    spacetime'' in \textit{The Oxford handbook of philosophy of
      physics} \rurl{doi.org/10.1093/oxfordhb/9780195392043.013.0016}
  \item Dasgupta. Substantivalism vs relationalism about space in
    classical physics \rurl{doi.org/10.1111/phc3.12219}
  \item Teitel. How to be a spacetime substantivalist
    \url{doi.org/10.5840/jphil2022119517}
  \item Nerlich. Space-time substantivalism
    \rurl{doi.org/10.1093/oxfordhb/9780199284221.003.0011}
  \end{itemize}

\item Revisionary relationalism

  There is a sub-branch of relationalism that aims to rewrite
  spacetime theories so that they do not ``quantify over spacetime
  points''. Similar thoughts might also have been the motivation for
  the so-called ``Einstein algebra program'' (see Earman, Weatherall
  et al.).

  \begin{itemize}
  \item Field. Can we dispense with space-time?
    \rurl{jstor.org/stable/192496}
  \item Manders. On the space-time ontology of physical theories
    \rurl{jstor.org/stable/187166}
  \item Babic and Cocco. Mandersian relationism: space, modality and
    equivalence \doi{10.1017/psa.2023.64}    
  \item Burgess. Sets and point-sets: five grades of set-theoretic
    involvement in geometry \rurl{jstor.org/stable/192905}
  \item Burgess. Synthetic mechanics \doi{10.1007/bf00247712}
  \item Burgess. Synthetic mechanics revisited
    \doi{10.1007/bf00284971}
  \item Bacon. Relative locations. \textit{Oxford Studies in
      Metaphysics}
  \item Hale. Spacetime and the abstract/concrete distinction
    \doi{10.1007/BF00355677}
  \end{itemize}

  The revisionary philosophers might also welcome the fact that
  General Relativity can be done without any spacetime points at all
  --- viz.\ via the the program of ``Einstein algebras''.

  \begin{itemize}
  \item Rosenstock et al. On Einstein algebras and relativistic
    spacetimes \doi{10.1016/j.shpsb.2015.09.003}
  \end{itemize}

  How are we supposed to think about the situation here? We have one
  theory formulation in which there are spacetime points, and one in
  which there are not. Moreover, these formulations are equivalent in
  a precise sense (as proven by Rosenstock et al.).

\item Incongruent counterparts

  Objects L and \reflectbox{L} are said to be \emph{incongruent
    counterparts} if: (a) no combination of translations and rotations
  can move L into the place of \reflectbox{L}, and (b) a combination
  of translations, rotations, and reflections \emph{can} move L into
  the place of \reflectbox{L}. Kant famously appealed to incongruent
  counterparts in two major arguments --- with seemingly contradictory
  conclusions. In the first, he argues that space is absolute. In the
  second, he argues that space is transcendentally ideal. Kant's
  former argument has been revived by substantivalists in recent
  years.

  \begin{itemize}
  \item Kant. Directions in space
  \item Kant. Prolegomena
  \item Nerlich. Hands, knees, and absolute space
  \item Earman. \emph{World enough and spacetime}, pp 137--153
  \item Hogan. Handedness, idealism and freedom
  \item Van Cleve and Frederick (eds.) \textit{The philosophy of right
      and left: incongruent counterparts and the nature of space}
  \item Pooley. Handedness, parity violation, and the reality of space
    [Relationalist counterargument: the existence of incongruent
    counterparts does not imply substantivalism]
  \item Nerlich. Incongruent counterparts and the reality of space
    \rurl{doi.org/10.1111/j.1747-9991.2009.00212.x}
%   \item Orientable manifolds
%   \item Knots 
  \end{itemize}

\item The hole argument
  % metric essentialism; counterpart theory
  \begin{itemize}
  \item Earman and Norton. What price spacetime substantivalism? The
    hole story \rurl{doi.org/10.1093/bjps/38.4.515}
  \item Butterfield. The hole truth \rurl{doi.org/10.1093/bjps/40.1.1}
  \item Maudlin. The essence of space-time \rurl{jstor.org/stable/192873}
  \item Maudlin. Substances and space-time: What Aristotle would have
    said to Einstein
  \item Rynasiewicz. Is there a syntactic solution to the hole
    problem?
  \item Stachel. The hole argument and some physical and philosophical
    implications \rurl{doi.org/10.12942/lrr-2014-1}
  \item Weatherall. Regarding the `hole argument'
    \rurl{doi.org/10.1093/bjps/axw012}
  \item Pooley and Read. On the mathematics and metaphysics of the
    hole argument \rurl{doi.org/10.1086/718274}
  \item Bradley and Weatherall. Mathematical responses to the hole
    argument: then and now \rurl{doi.org/10.1017/psa.2022.58}
  \item Bradley and Weatherall. On representational redundancy,
    surplus structure, and the hole argument
    \doi{10.1007/s10701-020-00330-y}
  \item Halvorson and Manchak. Closing the hole argument
    \rurl{doi.org/10.1086/719193}
  \item Mundy. Space-time and isomorphism
  \item Teitel. Holes in spacetime: Some neglected essentials
    \rurl{doi.org/10.5840/jphil2019116723}
  \item Jacobs. Some neglected possibilities: a reply to Teitel  
  \item Gomes and Butterfield. The hole argument and beyond, Part I:
    The story so far
  \item Gomes and Butterfield. The hole argument and beyond, Part II:
    Treating non-isomorphic spacetimes \doi{10.48550/arXiv.2303.14060}
  \item Dougherty. The hole argument, take $n$
    \doi{10.1007/s10701-019-00291-x}
  \item Ladyman and Presnell. The hole argument in homotopy-type
    theory \doi{10.1007/s10701-019-00293-9}
%% ETCS     
  \end{itemize}

  The hole argument is one of the best examples of a topic where
  physics, mathematics, and philosophy come together. Weatherall,
  Halvorson, and Manchak suggest that reasoning with the models of a
  theory never requires identification of points across models. Gomes
  and Butterfield disagree. Could we do all interesting mathematical
  physics in a structural set theory, such as Lawvere's ETCS? That
  question is interesting, not just for the hole argument, but for
  general questions about the indispensability of mathematical objects
  for physics. (My opinion is that physics can be systematically
  agnostic about most questions of mathematical ontology. If the hole
  problem depends crucially on Zermelo-Frankel set theory --- with its
  global elementhood relation --- then it's a pseudo-problem for
  physics.)

  The next two topics are central to the hole argument, but hold even
  more general philosophical interest.

\item Leibniz shift and possibility

  Question: if you shift a possible world one meter to the left then
  is the result a new possible world? i.e., do shifts create new
  possibilities? How do we count the number of possibilities? Does
  Ockham's razor apply in the space of possible worlds? Does the
  Principle of the Identity of Indiscernibles apply in the space of
  possible worlds? (My opinion is that we get confused when we try to
  reason about possible worlds as if they were concrete things.)

  \begin{itemize}
  \item Roberts. Regarding `Leibniz equivalence'
    \rurl{doi.org/10.1007/s10701-020-00325-9}
  \item Belot. Fifty million Elvis fans can't be wrong
  \item Halvorson. \emph{The logic in philosophy of science}, pp
    257--260
  \end{itemize}

\item Determinism
  \begin{itemize}
  \item Montague. ``Deterministic theories'' in Thomason (ed.)
    \textit{Formal Philosophy}
  \item Earman. \textit{Primer of determinism}
  \item Leeds. Holes and determinism: another look
    \rurl{doi.org/10.1086/289876}
  \item Belot. Determinism and ontology
    \rurl{doi.org/10.1080/02698599508573508}
  \item Brighouse. Determinism and modality
    \rurl{doi.org/10.1093/bjps/48.4.465}
  \item Melia. Holes, haecceitism and two conceptions of determinism 
%   \item Werndl. ``Deterministic versus indeterministic descriptions:
%    not that different after all?'' in Hieke and Leitgeb (eds.)
%     \textit{Reduction, Abstraction, Analysis}
%   \item M. Wilson, ``Determinism and the Mystery of the Missing Physics''
  \end{itemize}

\item Causality in relativity theory
  %% speed of light and signalling  
  \begin{itemize}
  \item Petzoldt. Kausalit{\"a}t und Relativit{\"a}tstheorie
  \end{itemize}

\item Spacetime representation \dots or, what actually is the aim of
  spacetime physics?
  \begin{itemize}
  \item Poincar{\'e}. On the foundations of geometry [Argues that
    there is no direct sense in which geometric spaces are supposed to
    represent physical space. But bases some of his arguments on
    suspect/falsified claims about human perception.]
  \item Galison. Minkowski space-time: from visual thinking to the
    absolute world \rurl{doi.org/10.2307/27757388}
  \item Mundy. The physical content of Minkowski geometry
    \rurl{jstor.org/stable/686996}
  \item Ludwig. Is the geometry of physical space a form of pure
    sensible intuition? a technical construction? or a structure of
    reality?  \rurl{dropbox.com/s/8x8z6ihfb99drg0/ludwig2.pdf?dl=0}
  \item Fletcher. On representational capacities, with an application
    to general relativity \rurl{doi.org/10.1007/s10701-018-0208-6}
  \item Wallace. Stating structural realism: mathematics-first
    approaches to physics and metaphysics
    \rurl{philsci-archive.pitt.edu/20048}
  \end{itemize}


\item Spacetime functionalism
  \begin{itemize}
  \item Baker. On spacetime functionalism
    \rurl{philpapers.org/rec/BAKOSF}
  \end{itemize}

\item Kant on space and time 
  
  The contemporary literature on space (and time) poses a dilemma:
  either realism or antirealism. We see the same dilemma in the
  Leibniz-Clarke correspondence. Kant's original answer was realism
  about space. But then something changed --- and he came to see the
  dilemma as false.
  \begin{itemize}
  \item Tolley. The difference between original, metaphysical, and
    geometrical representations of space
    \doi{10.1057/978-1-137-53517-7_11}
  \item Carrier. Kant's relational theory of absolute space
    \doi{10.1515/kant.1992.83.4.399}
  \item Hatfield. Kant on the perception of space (and time)
  \item Schneider. \textit{Das Raum-Zeit-Problem bei Kant und Einstein}
    \doi{10.1007/978-3-642-92225-1}
  \item Stan. Absolute space and the riddle of rotation: Kant’s
    response to Newton
  \end{itemize}

\item Space and time as conditions of things in themselves
  %% Maudlin (about quantum gravity? see Jaksland)
  % Maudlin. Completeness, supervenience, and ontology
  %% Esfeld 
  %% Carrier. Kant’s Relational Theory of Absolute Space 10.1515/kant.1992.83.4.399
  \begin{itemize}
  \item Hegel. \textit{Vorlesungen \"Uber die Geschichte der
      Philosophie, Band III} [\textit{Lectures on the history of
      philosophy} \rurl{gutenberg.org/ebooks/58169}]
  \item Inwood. ``Kant and Hegel on space and time'' in Priest (ed.)
    \textit{Hegel's Critique of Kant}
  \item Jenkins. Hegel on space: A critique of Kant's transcendental
    philosophy \doi{10.1080/0020174X.2010.493367}
    % F. A. Trendelenburg and the Neglected Alternative Andrew Specht
    % Esfeld. Super-Humeanism: The Canberra Plan for Physics
    %%%%% The Foundation of Reality: Fundamentality, Space, and Time
    %%%%% David Glick (ed.),  George Darby (ed.),  Anna Marmodoro (ed.)
  \end{itemize}
  
\end{enumerate}

\subsection*{Additional sources (mathematical foundations)}

\begin{itemize}
\item Naber. \emph{The geometry of Minkowski spacetime}
\item Mac Lane. \emph{Geometrical mechanics} [elegant presentation of
  differential geometry, but focused on analytic mechanics rather than
  relativity theory]
\item Moerdijk and Reyes. \textit{Models for smooth infinitesimal
    analysis}
\item Reyes. A derivation of Einstein's vacuum field equations
\item Ketland. Axiomatization of Galilean Spacetime  
\end{itemize}

\subsection*{Additional sources}

\begin{itemize}
\item Arntzenius. \textit{Space, time, and stuff}
\item Dorato. The affective and practical consequences of presentism
  and eternalism
\item Read. \textit{Special relativity}
\item Kragh. \textit{Quantum generations: a history of physics in the
    twentieth century}
\item Ismael. Rethinking time and determinism: what happens to
  determinism when you take relativity seriously
\item Halvorson. Objective description in physics
  \rurl{philpapers.org/rec/HALODI-2}
\item Disalle. Spacetime theory as physical geometry
  \doi{10.1007/bf01129008}
\item Dewar. General-relativistic covariance
  \doi{10.1007/s10701-019-00256-0}
\item Pais. \textit{`Subtle is the Lord \dots ': The science and the
    life of Albert Einstein}
\item Balashov. \emph{Persistence and spacetime}
\item Belot. \textit{Geometric possibility}  
\item Bridgman. \emph{A sophisticate's primer of relativity}
\item Darrigol. \textit{Relativity principles and theories from
    Galileo to Einstein}
\item Petzoldt. \emph{Die Stellung der Relativit{\"a}ts-Theorie in der
    geisteigen Entwicklung der Menschheit}
\item Russo-Krauss. \textit{The philosophy of Joseph Petzoldt: From
    Mach's positivism to Einstein's relativity}
\item Janssen. ``Special relativity'' in \emph{The Cambridge Companion
    to Einstein} \doi{10.1017/CCO9781139024525.017}
\item Landsman. \textit{Foundations of general relativity: from
    Einstein to black holes}
  \rurl{math.ru.nl/~landsman/FGRBook2022-online.pdf}
\item Norton. ``Philosophy of space and time'' in \emph{Introduction
    to the philosophy of science}
\item Norton. What we can learn about the ontology of space and time
  from the theory of relativity
  \doi{10.1093/acprof:oso/9780195145649.003.0008}
\item Norton. General covariance and the foundations of general
  relativity
\item Pesic. \textit{Beyond geometry}
\item Huggett. \emph{Space from Zeno to Einstein}
  \rurl{archive.org/details/B-001-001-239}
\item Schlick. \emph{Raum und Zeit in der gegenw{\"a}rtigen Physik}
\item Schlick. \textit{Texte zu Einsteins Relativit{\"a}tstheorie}
\item Cassirer. \textit{Zur Einsteinschen
    Relativitätstheorie. Erkenntnistheoretische Betrachtungen}
\item Knox and Wilson. \textit{The Routledge companion to philosophy
    of physics} (part II special relativity and part III general
  relativity)
\begin{itemize}
\item Brown and Read. ``The dynamical approach to spacetime theories''
\item Weatherall. ``Classical spacetime structure''
\end{itemize}
\item Fletcher. Relativistic spacetime structure
\item \textit{Toward a theory of spacetime theories}
  \doi{10.1007/978-1-4939-3210-8}
  \begin{itemize}
  \item Beisbart. A model-theoretic analysis of space-time theories
  \end{itemize}
  %% Beisbart on model theoretic
\item Sklar. \textit{Space, time, and spacetime}
\item Salmon. \textit{Space, time, and motion}
\item Torretti. \textit{Relativity and geometry}
\item Torretti. \textit{Philosophy of geometry from Riemann to
    Poincar{\'e}}
\item Torretti. Nineteenth century geometry
  \rurl{plato.stanford.edu/entries/geometry-19th}
\item Miller. \textit{Albert Einstein's special theory of relativity:
    emergence (1905) and early interpretation (1905-1911)}
\item E. Nagel. ``Relativity and twentieth-century intellectual life''
  in Woolf (ed.) \textit{Some strangeness in the proportion}
\item Craig and Smith (eds.) \textit{Einstein, relativity and absolute
    simultaneity}
\item Reichenbach. \textit{The philosophy of space and time}
\item Steane. \textit{Relativity made relatively easy}
  % Klaus Hentschel, Die Korrespondenz Petzold—Reichenbach: Zur Entwicklung der
  %% ‘wissenschaftlichen Philosophie’ in Berlin
\item Giovanelli. `But one must not legalize the mentioned sin':
  Phenomenological vs. dynamical treatments of rods and clocks in
  Einstein's thought \doi{10.1016/j.shpsb.2014.08.012}
\item Cutter. Spatial experience and special relativity
  \doi{10.1007/s11098-016-0799-8}
\item Peacock. A new look at simultaneity
  \rurl{jstor.org/stable/192782}
\item Belot. ``Time in classical and relativistic physics'' in Bardon
  and Dyke (eds.) \textit{A companion to the philosophy of time}
  %% \item Robb. \textit{Geometry of time and space}
\item Hiskies. Space-time theories and symmetry groups
  \doi{10.1007/BF00738921}
\item F{\o}llesdal. Relativity, rotation and rigidity
  % \item Slavov. ``Kaila's interpretation of Einstein-Minkowski
  %   invariance
  %   theory'' % https://doi.org/10.1016/j.shpsa.2022.03.001
\item Schilpp (ed.) \textit{Albert Einstein: Philosopher-Scientist}
  % Hatfield. ``Kant on the perception of space (and time)''
\item Hodgson. Relativity and religion. The abuse of Einstein's theory
  \doi{10.1111/1467-9744.00506}
\item Verhaegh. The reception of relativity in American
  philosophy. \doi{10.1017/psa.2023.85}
\item \fullcite{russo2023}
\item \fullcite{neuber2023}
\item \fullcite{meyerson}
\item \fullcite{holst}
\end{itemize}

% Nerlich, "Simultaneity and convention in special relativity"

% Russell. Is position in time and space absolute or relative?

% Jammer, Einstein and Religion (also talks about absolute
% simultaneity)

% Friedman, "Simultaneity in Newtonian Mechanics and Special
% Relativity''

% Craig, "God and real time"
% http://www.leaderu.com/offices/billcraig/docs/realtime.html

%% ". . . the theory of relativity put an end to the idea of absolute
%% time . . . . The theory of relativity does . . . force us to change
%% fundamentally our ideas of space and time" (Stephen Hawking, A
%% Brief History of Time [New York: Bantam Books, 1988], pp. 21, 23).

% Lawrence Sklar, "Time, reality and relativity," in Reduction, Time
% and Reality, ed. R. Healey (Cambridge: Cambridge University Press,
% 1981)

% G. Holton, "Mach, Einstein, and the Search for Reality," in Ernst
% Mach: Physicist and Philosopher

% Takeuchi. An Illustrated Guide to Relativity

%% Nijenhuis
%% natural bundles
%% Michior

%% Macfarlane. Assessment Sensitivity: Relative Truth and its Applications.

% Klaus Hentschel
  %% Die Korrespondenz Petzold—Reichenbach: Zur Entwicklung der
  %% ‘wissenschaftlichen Philosophie’ in Berlin

  % S. Wagner (1982). Ludwig Boltzmann and the special theory of
  % relativity, in Ludwig Boltzmann Gesammtausgabe: 8. Ausgewahlte
  % Abhandlungen der Internationalen Tagung Wien 1881, ed. R. Sexl and
  % J. Blackmore, pp. 341–54. Akademische Druck- und Verlagsanstalt,
  % Graz. (Boltzmann's use of the concept of reference frame)

  % D'Inverno
  % Wald
  % Lilian Lieber


\end{document}


%%% Local Variables:
%%% mode: latex
%%% TeX-master: t
%%% End:
